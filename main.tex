\documentclass{jsarticle}
\usepackage{amsmath}
\usepackage{amssymb}
\usepackage{amsfonts}
\usepackage{amsthm}
\usepackage{macros} 

\usepackage[backend=biber,style=alphabetic,url=false]{biblatex}
\addbibresource{main.bib}

\usepackage{makeidx}
\makeindex  

\title{算術的完全性定理}
\author{Mateusz Zunderewski}
\date{\today}

\begin{document}

\maketitle

\begin{abstract}
	この文書では主に $\LogicGL$ のSolovayの算術的完全性定理について述べる.
\end{abstract}

\setcounter{tocdepth}{3}
\tableofcontents

\pagebreak

\cite{数学基礎論増補版,数学における証明と真理}などを大いに参考にした.

\section{準備}

定義と,基本的な事実を確認する.

\begin{notation}
	自然数全体の集合は $\omega$ で表す.
\end{notation}

\subsection{不完全性定理}

\begin{notation}
	一般の算術の文については $\sigma, \pi, \dots$ のようにギリシャ文字で表すことにする.
\end{notation}

\begin{definition}
	$n \leq 1$ に対し,
	$\Provable_T^n(\ulcorner \sigma \urcorner) \equiv \underbrace{\Provable_T(\ulcorner \cdots \Provable_T(\ulcorner}_{n} \sigma \urcorner) \cdots \urcorner)$ と定義する.
\end{definition}


\subsection{様相論理}

\begin{notation}
	一般の様相論理式は $A,B,C,\dots$ のように普通のラテンアルファベットを用いて表すことにする.
\end{notation}

\begin{definition}
	$n \in \omega$ とする.$\Box^n A$ を再帰的に以下のように定める.
	\begin{itemize}
		\item $\Box^0 A \equiv A$
		\item $\Box^{n+1} A \equiv \Box \Box^n A$
	\end{itemize}
\end{definition}


\begin{definition}
	$F = \structure{W, R}$ をKripkeモデル,$x \in W$ とする.
	「$x$ から最大で何回遷移可能か」を表す順序数 $d(x)$ を以下のように定める.
	\begin{itemize}
		\item $d(x) = \max \{ n \in \omega \mid \exists y \in W \colon w R^n y \}$
		\item ただし $\{ n \in \omega \mid \exists y \in W \colon w R^n y \} = \omega$ のとき,$d(x) = \omega$ とする.
	\end{itemize}
\end{definition}

\begin{remark}
	定義より明らかだが,$d(x)$ が一般に自然数を取るとは限らない.
	例えば $W$ が無限であれば任意の $n \in \omega$ に対して $x R^n y$ となる $y$ を取ってこれるかもしれないし,
	仮に $W$ が有限だとしても $R$ が反射的ならば $x R x R x \cdots$ のようなループによって任意の $n \in \omega$ で $x R^n x$ かもしれない.

	ただし,特殊な構造を持つフレームに対しては $d(x)$ は各 $x$ で常に自然数となることが保証できる.
\end{remark}

何回遷移できるかが有限で抑えられるなら,次の補題\ref{lem:cannot_access}が成り立つ.

\begin{lemma}\label{lem:cannot_access}
	$d(x), n \in \omega$ とする.
	$d(x) \leq n$ ならば $M,x \vDash \Box^{n + 1} A$.
\end{lemma}

\begin{proof}
	厳密な証明は $n$ についての帰納法を用いて示せばよいが,
	直感的には $d(x) \leq n$ ということは $x R^{n + 1} y$ となる $y$ が存在しないということを考えれば明らか.
	\qed
\end{proof}

\subsubsection{$\LogicGL$ について}

\begin{definition}
	公理図式 $\AxL \equiv \Box(\Box A \to A) \to \Box A$ とする.\index{$\AxL$}
	$\LogicK \oplus \AxL$ を $\LogicGL$ と定義する.\index{$\LogicGL$}
\end{definition}

\begin{definition}
	有限 $\LogicGL$ 木モデル $M = (W = \{w_1,\dots,w_n\}, w_1, R, \Vdash)$ に対して,
	根から下に新しく $w_0$ を生やして拡張した有限木 $\LogicGL$ モデル $(y = \{w_0,w_1,\dots,w_n\}, w_0, R', \Vdash')$ を $M$ の単純拡張モデルと呼ぶ.

	よりきちんと定義すると以下のとおりで,これは有限木 $\LogicGL$ モデルとなる.
	\begin{itemize}
		\item $y = W \cup \{w_0\}$
		\item $R' = R \cup \{(w_0,w_i) \mid w_i \in W\}$
		\item $w \in W$ について $w \Vdash' p \iff w \Vdash p$
		\item $w_0 \Vdash' p \iff w_1 \Vdash p$
	\end{itemize}
\end{definition}

\begin{lemma}\label{lem:simple_extension}
	$M'$ は $M$ の単純拡張とする.$w_i \in W$ ならば $M,w_i \nvDash B \implies M',w_i \nvDash B$
\end{lemma}
\begin{proof}
	新しく追加された $w_0$ 以外は元のモデルと同じであり,$w_i \in W \iff w_i \neq w_0$ であるから,ほとんど明らか.
	\qed
\end{proof}

\subsubsection{$\LogicGrz$ について}

\begin{definition}
	公理図式 $\AxGrz \equiv \Box(\Box(A \to \Box A) \to A) \to A$ \index{$\AxGrz$} とする.
	$\LogicK \oplus \AxGrz$ を $\LogicGrz$ と定義する.\index{$\LogicGrz$}
\end{definition}

\begin{lemma}
	$\LogicGrz \vdash \AxT$ および,$\LogicGrz \vdash \AxFour$.
\end{lemma}

よって直ちに次が成り立つ.

\begin{corollary}\label{cor:S4_weakerThan_Grz}
	$\LogicSFour \leq \LogicGrz$.
	また $\LogicSFourGrz = \LogicSFour \oplus \AxGrz$ として $\LogicSFourGrz = \LogicGrz$.
\end{corollary}

\begin{remark}
	$\LogicSFour$ およびその拡張に対して $\AxGrz$ という公理を追加した論理は,
	\ref{sect:modal_copanion}節で考えるModal Companionという概念で重要となる.
\end{remark}

\begin{lemma}
	$F \vDash \AxGrz$ ならば $F \vDash \AxT$ かつ $F \vDash \AxFour$ である.
\end{lemma}

したがって $\AxT$ と $\AxFour$ のフレーム定義性より,直ちに次の系が得られる.

\begin{corollary}
	$F \vDash \AxGrz$ ならば $F$ は反射的かつ推移的.
\end{corollary}

\begin{definition}
	$\Logic{\Lambda}$ は様相論理とし,$\FrameClass$ は空でないフレームクラスとする.
	\begin{enumerate}
		\item
		      任意の $F$ で $F \in \FrameClass \implies F \in \FrameClassOf{\Logic{\Lambda}} $ のとき,
		      $\FrameClassOf{\Logic{\Lambda}}$ は $\FrameClass$ によって特徴づけられるという.
		\item
		      任意の $F$ で $F \in \FrameClass \iff F \in \FrameClassOf{\Logic{\Lambda}} $ のとき,
		      $\FrameClassOf{\Logic{\Lambda}}$ は $\FrameClass$ によって定義されるという.
	\end{enumerate}
\end{definition}

\begin{corollary} \label{cor:LogicGrz_definedBy}
	$\FrameClassOf{\LogicGrz}$ は反射的,推移的,弱逆整礎的なフレームのクラスによって定義される.
\end{corollary}

\begin{lemma}\label{lem:wcwf_of_finite_trans_antisymm}
	$W$ 上の2項関係 $R$ について,$W$ が有限かつ $R$ が推移的かつ反対称的ならば $W$ は弱逆整礎的である.
\end{lemma}

系\ref{cor:LogicGrz_definedBy}と補題\ref{lem:wcwf_of_finite_trans_antisymm}より,$\LogicGL$ のときと同様に使いやすい次の系が得られる.
\begin{corollary}
	$\FiniteFrameClassOf{\LogicGrz}$ は反射的,推移的,反対称的な有限フレームクラスによって定義される.
\end{corollary}

\subsubsection{$\LogicGrz$ のKripke完全性}

基本的には $\LogicGL$ のKripke完全性証明と同じように行い,コンプリートモデルの構成も同様に行う.

\begin{definition}
	有限集合 $\GrzSub{A} := \Sub{A} \cup \{ \Box(\Box \to \Box A) \mid A \in \Sub{A} \}$ と定義する.
\end{definition}

\begin{definition}
	$X, S$ は有限集合とする.
	任意の $A \in S$ に対して $A \in X$ または $-A \in X$ であるとき,$X$ が $S$ に対して補で閉じているという.

	$X$ が $S$ に対して補で閉じて $\Logic{\Lambda}$ 無矛盾な論理式の有限集合であるとき,簡潔にCCF\index{CCF}と呼ぶことにする.
\end{definition}

\begin{lemma}
	$S$ に対してのCCFはたかだか有限個しか存在しない.
\end{lemma}

$\LogicGL$ と同じように,CCFによって有限コンプリートモデルを構成する.
\begin{definition}
	$A$ に対しての $\LogicGrz$ 有限コンプリートフレーム $F_{\LogicGrz, A} := \structure{W_{\LogicGrz, A}, R_{\LogicGrz, A}}$ を以下のように定義する.
\end{definition}

\begin{lemma}
	$F_{\LogicGrz, A}$ は反射的,推移的,反対称的.
\end{lemma}

\begin{definition}
	$M_{\LogicGrz, A} := \structure{F_{\LogicGrz, A}, \Vdash_{\LogicGrz, A}}$ を $A$ に対しての $\LogicGrz$ 有限コンプリートモデルという.
	ただし,$w \Vdash_{\LogicGrz, A} p \iff p \in W$ とする.
\end{definition}

やはり $\LogicGL$ と同様に真理補題が成り立つ.その前にいくつかの補題を示す.

\begin{lemma}\label{lem:Grz_truthlemma}
	$B \in \Sub{A}$ とする.
	$M_{\LogicGrz, A}, X \vDash B \iff B \in X$.
\end{lemma}

ここまでくれば,あとは $\LogicGL$ と同じように完全性を示すことが出来る.

\begin{theorem}
	$\LogicGrz$ は反射的,推移的,反対称的な有限フレームクラスに対してKripke完全.
\end{theorem}

よって直ちに次のこともわかる.

\begin{corollary}
	$\LogicGrz$ は有限フレーム性を持ち,したがって決定可能.
\end{corollary}

\subsubsection{Boxdot Companion}

\begin{definition}
	$\boxdot A \equiv A \land \Box A$ とする.
\end{definition}

\begin{definition}[Boxdot変換]
	$A^\boxdot$ は $A$ の中に現れる $\Box$ を全て $\boxdot$ で置き換えた論理式とし,その変換 $\cdot^\boxdot$ をBoxdot変換\index{Boxdot変換}と呼ぶ.
	もう少し厳密に書けば,以下のように再帰的に定義される.
\end{definition}

\begin{definition}
	様相論理 $\Logic{\Lambda_1}$ が $\Logic{\Lambda_2}$ のBoxdot Companionであるとは,
	任意の様相論理式 $A$ に対して $\Logic{\Lambda_1} \vdash A \iff \Logic{\Lambda_2} \vdash A^\boxdot$ が成り立つことである.
\end{definition}

本節では証明可能性論理という面でやや重要な事実である,$\LogicGrz$ が $\LogicGL$ のBoxdot Companionであること(定理\ref{thm:Grz_GL_BoxdotCompanion})を示す.
なお,他の論理に対するもっと一般的な結果が\cite*{jerabek_cluster_2016}によって示されている.

片側である補題\ref{lem:Grz_GL_BoxdotCompanion_1}はKripke意味論を用いて示す.

\begin{lemma}\label{lem:Grz_GL_BoxdotCompanion_1}
	$\LogicGL \vdash A^\boxdot \implies \LogicGrz \vdash A$
\end{lemma}

\begin{definition}
	$F = \structure{W, R}$ はKripkeフレームとする.
	\begin{enumerate}
		\item $F$ の反射化\index{はんしゃか@反射化} $\reflexv{F} := R \cup \{ (x, x) \in x \in W\}$ とする.
		\item $F$ の非反射化\index{ひはんしゃか@非反射化} $\irreflxv{F} := \{(x,y) \in R \mid x \neq y \}$ とする.
	\end{enumerate}
\end{definition}

\begin{lemma}
	$F = \structure{W, R}$ をKripkeフレーム,$\Vdash$ を付値,$x \in W$,$A$ は任意の様相論理式とする.

	$\structure{F, \Vdash}, x \vDash A^\boxdot \iff \structure{\reflexv{F}, \Vdash}, x \vDash A$
\end{lemma}

\begin{proof}
	$A$ の構造に対する帰納法で示す.自明ではないのは $A \equiv \Box B$ のときだけなのでそれだけ示せば十分.
\end{proof}

\begin{corollary}
	$F$ をKripkeフレーム,$A$ は任意の様相論理式とすれば $F \vDash A^\boxdot \iff \reflexv{F} \vDash A^\boxdot$
\end{corollary}

\begin{lemma}
	$F = \structure{W, R}$ を反射的なKripkeフレーム,$\Vdash$ を付値,$x \in W$,$A$ は任意の様相論理式とする.
	$\structure{F,\Vdash}, x \vDash A \iff \structure{\reflexv{(\irreflxv{F})}, \Vdash}, x \vDash A$.
\end{lemma}


もう片側である補題\ref{lem:Grz_GL_BoxdotCompanion_2}もKripke意味論を用いた議論で示すことが出来るが,
こちらには構文論的な議論のみで示す非常にパズル的な証明\cite[pp163-164]{boolos_logic_1994}が知られている.

\begin{lemma}\label{lem:Grz_GL_BoxdotCompanion_2}
	$\LogicGrz \vdash A \implies \LogicGL \vdash A^\boxdot$
\end{lemma}


こうして補題\ref{lem:Grz_GL_BoxdotCompanion_1}と補題\ref{lem:Grz_GL_BoxdotCompanion_2}より,
直ちに $\LogicGrz$ が $\LogicGL$ のBoxdot Companionであることがわかる.

\begin{theorem}\label{thm:Grz_GL_BoxdotCompanion}
	$\LogicGrz$ は $\LogicGL$ のBoxdot Companion である.
\end{theorem}


\subsubsection{Modal Companion}\label{sect:modal_copanion}

この節では公理 $\AxGrz$ のもう一つの側面であるModal Companionという概念について述べる.
Modal Companionのより発展的な話題については\cite*{chagrov_modal_1992}を参照されたい.

\begin{definition}[Gödel変換]\index{げーでるへんかん@Gödel変換}
	命題論理式を様相論理式へ写すGödel変換 $\cdot^G$ という操作を以下のように定義する.
\end{definition}

直観主義命題論理 $\LogicInt$ と様相論理 $\LogicSFour$ について次の定理\ref{thm:GMT}が知られている.

\begin{theorem}[Gödel-McKinsey-Tarskiの定理\cite*{godel_interpretation_1933,mckinsey_theorems_1948}]\label{thm:GMT}
	\begin{equation*}
		\LogicInt \vdash A \iff \LogicSFour \vdash A^G
	\end{equation*}
\end{theorem}

定理\ref{thm:GMT}をより一般化した関係を考える.

\begin{definition}[中間論理]\index{ちゅうかんろんり@中間論理}
	直観主義命題論理 $\LogicInt$ と古典命題論理 $\LogicCl$ の間の強さの論理を中間論理と呼ぶ.
	ただし $\LogicInt$ と $\LogicCl$ も中間論理に含める.
	すなわち,命題論理 $\Logic{I\Lambda}$ が中間論理とは $\LogicInt \leq \Logic{I\Lambda} \leq \LogicCl$ が成り立つことである.
\end{definition}

\begin{definition}[Modal Companion]\index{Modal Companion}
	様相論理 $\Logic{M\Lambda}$ が中間論理 $\Logic{I\Lambda}$ のModal Companionであるとは,
	任意の様相論理式 $A$ に対して以下が成り立つこととする.
	\begin{equation*}
		\Logic{I\Lambda} \vdash A \iff \Logic{M\Lambda} \vdash A^G
	\end{equation*}
\end{definition}

この定義よりGödel-McKensey-Tarskiの定理は「$\LogicSFour$ は $\LogicInt$ のModal Companionである」とも言い換えられる.


系\ref{cor:S4_weakerThan_Grz}より $\LogicSFourGrz \vdash A \iff \LogicGrz \vdash A \implies \LogicSFour \vdash A$ であるから,
$\LogicSFourGrz \vdash A^G \implies \LogicInt \vdash A$ が成り立つ.
実はこの逆も成り立つことがわかっているため,次のことが成り立つ.

\begin{theorem}[\cite*{grzegorczyk_relational_1969}]
	$\LogicSFourGrz$ も $\LogicInt$ のModal Companionである.
\end{theorem}

よって $\LogicInt$ のModal Companionとなる様相論理は複数存在する.
そしてこれは $\LogicInt$ だけに限らず一般の中間論理に対しても同様に,なおかつ次の強い主張が成り立つことも知られている.

\begin{theorem}[\cite*{maksimova_lattice_1974}]
	任意の中間論理 $\Logic{I\Lambda}$ のModal Companionは無限個存在し,かつ $\leq$ によって束を形成する.
\end{theorem}

そのような束の下限と上限,つまり各論理のModal Companionの中で最弱と最強の論理を考えたい.

\begin{definition}
	中間論理 $\Logic{I\Lambda}$ とし,その任意のModal Companionを $\Logic{M\Lambda}$ とする.

	\begin{enumerate}
		\item
		      $\tau\Logic{I\Lambda} \leq \Logic{M\Lambda}$ が成り立つような
		      最弱の $\Logic{I\Lambda}$ のModal Companionを $\tau\Logic{I\Lambda}$ とする.
		\item
		      $\Logic{M\Lambda} \leq \sigma\Logic{I\Lambda}$ が成り立つような
		      最強の $\Logic{I\Lambda}$ のModal Companionを $\sigma\Logic{I\Lambda}$ とする.
	\end{enumerate}
\end{definition}

\begin{theorem}[\cite*{dummett_modal_1959}]
	中間論理 $\Logic{I\Lambda}$ に対して,$\tau \Logic{I\Lambda} = \LogicSFour \oplus \{ A^G \mid \Logic{I\Lambda} \vdash A \} $ である.
\end{theorem}

\begin{corollary}
	$\tau \LogicInt = \LogicSFour$ である.
\end{corollary}

\begin{theorem}[Blok-Esakiaの定理\cite*{blok_varieties_1976,esakia_theory_1979,esakia_varieties_1979}]
	中間論理 $\Logic{I\Lambda}$ に対して,$\sigma \Logic{I\Lambda} = \tau\Logic{I\Lambda} \oplus \AxGrz$ である.
\end{theorem}

\begin{corollary}
	$\sigma \LogicInt = \LogicSFourGrz$ である.
\end{corollary}


\subsection{算術的意味論}

\begin{remark}
	以下,理論といえば算術の理論を指すことにする.
\end{remark}

\begin{definition}
	様相論理の命題変数から算術の文への写像 $* \colon \Prop_{\LangM} \to \Sent_{\LangA}$ を
	算術的解釈\index{さんじゅつてきかいしゃく@算術的解釈}あるいは単に解釈\index{かいしゃく@解釈}と呼ぶ.
\end{definition}

\begin{definition}
	解釈 $f$ と理論 $T$ の証明可能性述語 $\Provable_T$ に対して,
	以下のように定義域が様相論理式一般に拡張された写像 $*_{\Provable_T} \colon \Form_{\LangM} \to \Sent_{\LangA}$ を
	算術的変換\index{さんじゅつてきへんかん@算術的変換}あるいは単に変換\index{へんかん@変換}と呼ぶ.
	ただし読みづらいため,文脈上 $\Provable_T$ が明らかな場合は解釈の記号と同様に $*$ として記号を濫用することにする.
\end{definition}

\begin{example}
	第2不完全性定理で用いられる無矛盾性を表す文 $\mathrm{Con}_T$ を $\lnot \Provable_T(\ulcorner \bot \urcorner)$ として定義したことを思い出せば,
	この文は様相論理式 $\lnot \Box \bot$ を適当な変換 $*$ を用いて解釈した文として表すことが出来る.
	すなわち,$\mathrm{Con}_T \equiv (\lnot \Box \bot)^{*_{\Provable_T}}$ である.
\end{example}



\section{$\LogicGL$ の算術的完全性定理}

Solovay\cite{solovay_provability_1976}は証明可能性論理の出発点である次の定理を示した.この章ではその証明を追っていく.

\begin{theorem}[$\LogicGL$ の算術的完全性定理 \cite*{solovay_provability_1976}]\label{thm:GL_arith_completeness}
	$\Provable_T$ は $T$ の標準的な証明可能性とし,
	$A$ は任意の様相論理式とする.

	任意の解釈 $*$ に対して $T \vdash A^{*_{\Provable_T}}$ であるなら,$\LogicGL \vdash A$ である.
\end{theorem}

\begin{proof}[定理\ref{thm:GL_arith_completeness}の証明(1/3)]
	対偶を示す.すなわち,$\LogicGL \nvdash A$ であるなら,ある解釈 $*$ が存在して $T \nvdash A^*$ であることを示す.

	$\LogicGL \nvdash A$ であるから完全性より,
	ある有限木モデル $M = (W = \{w_1, \dots, w_n\}, w_1, R, \Vdash)$ が存在して $M,w_1 \nvDash A$ である.
	ただし今後,$W$ を自然数上のものとすると取り扱いやすい.
	すなわち,$M = (W = \{1,\dots,n\}, 1, R, \Vdash)$ とし,$M,1 \nvDash A$ として考える.
\end{proof}


\begin{definition}
	$M = (W = \{0,1,\dots,n\}, 0, R, \Vdash)$ に対し,以下の4条件を満たす $n + 1$ 個の文の列 $\beta_0, \beta_1, \dots, \beta_n$ を $M$ のSolovay文の列と呼ぶ.
	\begin{enumerate}
		\item $T \vdash \bigvee_{i \in W} \beta_i$
		\item $i \neq j$ ならば $T \vdash \beta_i \to \lnot \beta_j$
		\item $i \neq 0$ ならば $T \vdash \beta_i \to \Provable_T(\ulcorner \bigvee_{i R j} \beta_j \urcorner)$
		\item $i R j$ ならば $T \vdash \beta_i \to \lnot \Provable_T(\ulcorner \lnot \beta_j \urcorner)$
	\end{enumerate}
\end{definition}

\begin{remark}
	Solovay文の列が構成出来るかどうかは全く自明ではないが,一旦は構成可能であるとする.
	このことは後で確かめる.
\end{remark}

所望の解釈が次である.

\begin{definition}
	$\beta_0, \beta_1, \dots, \beta_n$ を $M$ のSolovay文の列とする.
	$p^* \equiv \bigvee_{\substack{i \in W \\ M, i \Vdash p}} \beta_i$ として定義される解釈 $*$ をSolovay解釈と呼ぶ.
\end{definition}

Solovay解釈の性質として重要なのが次の補題である.

\begin{lemma}\label{lem:solovay_interpretation}
	$*$ はSolovay解釈とする.任意の $i \neq 0$ と様相論理式 $B$ に対して次が成り立つ.
	\begin{enumerate}
		\item $M, i \vDash B \implies T \vdash \beta_i \to B^*$
		\item $M, i \nvDash B \implies T \vdash \beta_i \to \lnot B^*$
	\end{enumerate}
\end{lemma}
\begin{proof}
	$B$ の構成についての帰納法で同時に示す.
	\begin{itemize}
		\item
		      $B \equiv p$ とする.
		      \begin{enumerate}
			      \item
			            $i \vDash p$ のとき,$T \vdash \beta_i \to \bigvee_{j \Vdash p} \beta_j $ を示せば良いが,これは明らか.
			      \item
			            $i \nvDash p$ のとき $T \vdash \beta_i \to \lnot \bigvee_{j \Vdash p} \beta_j$ を示す.
			            これは $T \vdash \beta_i \to \bigwedge_{j \Vdash p} \lnot \beta_j$ と同値である.
		      \end{enumerate}
		\item $B \equiv \bot$ のときは明らか.
		\item $B \equiv C \to D$ のときは帰納法の仮定を用いて明らか.
		\item
		      $B \equiv \Box C$ のとき.
		      \begin{enumerate}
			      \item
			            $i \vDash \Box C$ のとき,任意の $i R j$ な $j$ で $M, j \vDash C$ である.
			            帰納法の仮定より $T \vdash \beta_j \to C^*$ であるので,$T \vdash \bigvee_{i R j} \beta_j \to C^*$.
			            D1とD2より $T \vdash \Provable(\ulcorner \bigvee_{i R j} \beta_j \urcorner) \to \Provable(\ulcorner C^* \urcorner)$.
			            S3より $T \vdash \beta_i \to \Provable_T(\ulcorner \bigvee_{i R j} \beta_j \urcorner)$ だったから,
			            合わせて $T \vdash \beta_i \to \Provable_T(\ulcorner C^* \urcorner)$ を得る.
			            これは $T \vdash \beta_i \to (\Box C)^*$ である.
			      \item
			            $i \nvDash \Box C$ のとき,ある $i R j$ な $j$ で $M, j \vDash C$ である.
			            帰納法の仮定より $T \vdash \beta_j \to \lnot C^*$ であり,対偶を取れば $T \vdash C^* \to \lnot \beta_j$ である.
			            D1とD2より $T \vdash \Provable_T(\ulcorner C^* \urcorner) \to \Provable_T(\ulcorner \lnot \beta_j \urcorner)$.
			            ここでS4の対偶より $T \vdash \Provable_T(\ulcorner \lnot \beta_j \urcorner) \to \lnot \beta_i$ であるから,
			            合わせて $T \vdash \Provable_T(\ulcorner C^* \urcorner) \to \lnot \beta_i$ を得る.
			            これは $T \vdash (\Box C)^* \to \lnot \beta_i$ であり,対偶を取れば $T \vdash \beta_i \to \lnot (\Box C)^*$ となる.
		      \end{enumerate}
	\end{itemize}
	\qed
\end{proof}

さて定理\ref{thm:GL_arith_completeness}の証明が出来る.

\begin{proof}[定理\ref{thm:GL_arith_completeness}の証明(2/3)]
	$M,1 \nvDash A$ だった.$M$ の単純拡張モデル $M' = \structure{y, R', \Vdash'}$ とすると補題\ref{lem:simple_extension}より $M',1 \nvDash A$ でもある.
	$M'$ に対してのSolovay文 $\beta_0, \beta_1, \dots, \beta_n$ を構成し,$*$ はSolovay解釈とする.
	補題\ref{lem:solovay_interpretation}より $T \vdash \beta_1 \to \lnot A^*$ を得て,対偶を取れば $T \vdash A^* \to \beta_1$ が得られる.

	今,仮に $T \vdash A^*$ と仮定すると議論が破綻することを示す.

	先ほどと合わせると $T \vdash \beta_1$ であり,更に導出可能性条件 D1 より $T \vdash \Provable_T(\ulcorner \beta_1 \urcorner)$ が得られる.
	一方,$M$ が単純拡張モデルなので $0 R' 1$ であるからS4より $T \vdash \beta_0 \to \lnot \Provable_T(\ulcorner \beta_1 \urcorner)$ であり,
	対偶を取れば $T \vdash \Provable_T(\ulcorner \beta_1 \urcorner) \to \lnot \beta_0$ が得られる.
	これらを合わせると $T \vdash \lnot \beta_0$ が得られる.
	したがってS1より $T \vdash \bigvee_{i \in W \setminus \{0\}} \beta_i$ である.

	$i \in y$ かつ $i \neq 0$ としたとき,$d(i) \leq d(1)$ である.したがって補題 \ref{lem:cannot_access}より
	$M',i \vDash \Box^{d(1) + 1} \bot$ が得られる.
	更に補題\ref{lem:solovay_interpretation}より $T \vdash \beta_i \to (\Box^{d(1) + 1} \bot)^*$ であり,
	これはすなわち $T \vdash \beta_{i} \to \Provable^{d(1) + 1}_T(\ulcorner \bot \urcorner)$ である.

	$T \vdash \bigvee_{i \in W \setminus \{0\}} \beta_i$ と合わせると $T \vdash \Provable^{d(1) + 1}_T(\ulcorner \bot \urcorner)$ を得る.
	$\Provable^{d(1) + 1}_T(\ulcorner \bot \urcorner)$ は $\Sigma_1$ 文であるから,$\Sigma_1$ 健全性より $T \vdash \bot$ が言える.
	しかしこれはおかしい.よって仮定がおかしく,$T \nvdash A^*$ である.
	\qed
\end{proof}

残ったのはSolovay文の列が構成可能であることを示すことである.

\subsection{Solovay文の列の構成}

\printbibliography

\printindex

\end{document}