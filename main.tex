\documentclass{jsarticle}
\usepackage{amsmath}
\usepackage{amssymb}
\usepackage{amsfonts}
\usepackage{amsthm}
\usepackage[backend=biber]{biblatex}
\addbibresource{main.bib}

\newcommand*{\Lang}[1]{\mathcal{L}_\mathrm{#1}}
\newcommand*{\LangA}{\Lang{A}}
\newcommand*{\LangML}{\Lang{ML}}

\newcommand*{\Logic}[1]{\mathbf{#1}}
\newcommand*{\LogicGL}{\Logic{GL}}

\newcommand*{\Prop}{\mathrm{Prop}}
\newcommand*{\MF}{\mathrm{MF}}
\newcommand*{\Provable}{\mathrm{Pr}}
 
\theoremstyle{definition}
\newtheorem{theorem}{定理}[section]
\newtheorem{lemma}[theorem]{補題}
\newtheorem{definition}[theorem]{定義}
\newtheorem{remark}[theorem]{注意}
\newtheorem{example}[theorem]{例}

\newtheorem*{theorem*}{定理}
\newtheorem*{lemma*}{補題}

\renewcommand{\proofname}{証明}
\makeatletter
\renewenvironment{proof}[1][\proofname]{\par
    \normalfont 
    \topsep6\p@\@plus6\p@\relax
    \trivlist
    \item\relax
    % {\itshape
    {\bfseries\gtfamily
    #1\@addpunct{.}}\hspace\labelsep\ignorespaces
    }{%
    \endtrivlist
    \@endpefalse
}
\makeatother


\title{算術的完全性定理}
\author{Mateusz Zunderewski}
\date{\today}

\begin{document}

\maketitle

\begin{abstract}
    この文書では主に $\LogicGL$ のSolovayの算術的完全性定理について述べる.
\end{abstract}

\section{準備}

定義と,基本的な事実を確認する.

\subsection{不完全性定理}

\begin{remark}
    一般の算術の文については $\sigma, \pi, \dots$ のようにギリシャ文字で表すことにする.
\end{remark}

\begin{definition}
    $n \leq 1$ に対し,
    $\Provable_T^n(\ulcorner \sigma \urcorner) \equiv \underbrace{\Provable_T(\ulcorner \cdots \Provable_T(\ulcorner}_{n} \sigma \urcorner) \cdots \urcorner)$ と定義する.
\end{definition}


\subsection{様相論理}

\begin{remark}
    一般の様相論理式は $A,B,C,\dots$ のように普通のラテンアルファベットを用いて表すことにする.
\end{remark}

\begin{definition}
    $\Box^n A$ を再帰的に以下のように定める.
    \begin{itemize}
        \item $\Box^0 A \equiv A$
        \item $\Box^{n+1} A \equiv \Box \Box^n A$
    \end{itemize}
\end{definition}


\begin{definition}
    $M = (W, R, \Vdash)$ をKripkeモデル,$w \in W$ とする.
    $w$ を始点とした $w R w_1 R w_2 \cdots$ という列で最長な列の長さを $d(w)$ とする.
\end{definition}

\begin{remark}
    一般に $d(w)$ は自然数を取るとは限らない.
    例えば $R$ が反射的ならば,$w R w_0 R w_1 R w_0 \cdots$ として任意の長さの列が存在しうるため,$d(w) = \infty$ となりうる.

    しかし,$\LogicGL$ 有限木モデルは非反射的なのでそのような場合は考えなくてよく,そのため $d(w)$ は有限として考えてよい.
    以下では $d(w) \in \omega$ とする.
\end{remark}

\begin{lemma}\label{lem:cannot_access}
    $d(w) \leq n$ ならば $M,w \vDash \Box^{n + 1} A$
\end{lemma}

\begin{proof}
    直感的には $d(w) \leq n$ ということは $w R^{n + 1} w'$ となる $w'$ が存在しないということを踏まえればほとんど明らか.
    厳密な証明は $n$ についての帰納法を用いて示せばよい.
    \qed
\end{proof}

\subsection{$\LogicGL$ について}

\begin{definition}
    有限 $\LogicGL$ 木モデル $M = (W = \{w_1,\dots,w_n\}, w_1, R, \Vdash)$ に対して,
    根から下に新しく $w_0$ を生やして拡張した有限木 $\LogicGL$ モデル $(W' = \{w_0,w_1,\dots,w_n\}, w_0, R', \Vdash')$ を $M$ の単純拡張モデルと呼ぶ.

    よりきちんと定義すると以下のとおりで,これは有限木 $\LogicGL$ モデルとなる.
    \begin{itemize}
        \item $W' = W \cup \{w_0\}$
        \item $R' = R \cup \{(w_0,w_i) \mid w_i \in W\}$
        \item $w \in W$ について $w \Vdash' p \iff w \Vdash p$
        \item $w_0 \Vdash' p \iff w_1 \Vdash p$
    \end{itemize}
\end{definition}

\begin{lemma}\label{lem:simple_extension}
    $M'$ は $M$ の単純拡張とする.$w_i \in W$ ならば $M,w_i \nvDash B \implies M',w_i \nvDash B$
\end{lemma}
\begin{proof}
    新しく追加された $w_0$ 以外は元のモデルと同じであり,$w_i \in W \iff w_i \neq w_0$ であるから,ほとんど明らか.
    \qed
\end{proof}

\subsection{算術的解釈}

\begin{remark}
    以下,理論といえば算術の理論を指すことにする.
\end{remark}

\begin{definition}
    命題変数から算術の文への写像 $* \colon \Prop \to \LangA$ を解釈と呼ぶ.
\end{definition}

\begin{definition}
    解釈 $f$ と理論 $T$ の証明可能性述語 $\Provable_T$ に対して,以下のように定義域が様相論理式一般に拡張される写像 $*_{\Provable_T} \colon \MF \to \LangA$ を変換と呼ぶ.
    ただし読みづらいため,文脈上 $\Provable_T$ が明らかな場合は解釈の記号と同様に $*$ として記号を濫用することにする.
\end{definition}

\begin{example}
    第2不完全性定理で用いられる無矛盾性を表す文 $\mathrm{Con}_T$ を $\lnot \Provable_T(\ulcorner \bot \urcorner)$ として定義したことを思い出せば,
    この文は様相論理式 $\lnot \Box \bot$ を適当な変換 $*$ を用いて解釈した文として表すことが出来る.
    すなわち,$\mathrm{Con}_T \equiv (\lnot \Box \bot)^{*_{\Provable_T}}$ である.
\end{example}



\section{$\LogicGL$ の算術的完全性定理}

Solovayは証明可能性論理の出発点である次の定理を示した.この節ではその証明を追っていく.

\begin{theorem}[$\LogicGL$ の算術的完全性定理 \cite*{solovay_provability_1976}]\label{thm:GL_arith_completeness}
    $\Provable_T$ は $T$ の標準的な証明可能性とし,
    $A$ は任意の様相論理式とする.

    任意の解釈 $*$ に対して $T \vdash A^{*_{\Provable_T}}$ であるなら,$\LogicGL \vdash A$ である.
\end{theorem}

\begin{proof}[定理\ref{thm:GL_arith_completeness}の証明(1/3)]
    対偶を示す.すなわち,$\LogicGL \nvdash A$ であるなら,ある解釈 $*$ が存在して $T \nvdash A^*$ であることを示す.

    $\LogicGL \nvdash A$ であるから完全性より,
    ある有限木モデル $M = (W = \{w_1, \dots, w_n\}, w_1, R, \Vdash)$ が存在して $M,w_1 \nvDash A$ である.
    ただし今後,$W$ を自然数上のものとすると取り扱いやすい.
    すなわち,$M = (W = \{1,\dots,n\}, 1, R, \Vdash)$ とし,$M,1 \nvDash A$ として考える.
\end{proof}


\begin{definition}
    $M = (W = \{0,1,\dots,n\}, 0, R, \Vdash)$ に対し,以下の4条件を満たす $n + 1$ 個の文の列 $\beta_0, \beta_1, \dots, \beta_n$ を $M$ のSolovay文の列と呼ぶ.
    \begin{itemize}
        \item $T \vdash \bigvee_{i \in W} \beta_i$
        \item $i \neq j$ ならば $T \vdash \beta_i \to \lnot \beta_j$
        \item $i \neq 0$ ならば $T \vdash \beta_i \to \Provable_T(\ulcorner \bigvee_{i R j} \beta_j \urcorner)$
        \item $i R j$ ならば $T \vdash \beta_i \to \lnot \Provable_T(\ulcorner \lnot \beta_j \urcorner)$ 
    \end{itemize}
\end{definition}

\begin{remark}
    Solovay文の列が構成出来るかどうかは全く自明ではないが,一旦は構成可能であるとする.
    このことは後で確かめる.
\end{remark}

所望の解釈が次である.

\begin{definition}
    $\beta_0, \beta_1, \dots, \beta_n$ を $M$ のSolovay文の列とする.
    $p^* \equiv \bigvee_{i \in W \\ M, i \Vdash p} \beta_i$ として定義される解釈 $*$ をSolovay解釈と呼ぶ.
\end{definition}

Solovay解釈の性質として重要なのが次の補題である.

\begin{lemma}\label{lem:solovay_interpretation}
    $*$ はSolovay解釈とする.任意の $i \neq 0$ と様相論理式 $B$ に対して次が成り立つ.
    \begin{enumerate}
        \item $M, i \vDash B \implies T \vdash \beta_i \to B^*$
        \item $M, i \nvDash B \implies T \vdash \beta_i \to \lnot B^*$
    \end{enumerate}
\end{lemma}

さて定理\ref{thm:GL_arith_completeness}の証明が出来る

\begin{proof}[定理\ref{thm:GL_arith_completeness}の証明(2/3)]
    $M,1 \nvDash A$ だった.$M$ の単純拡張モデル $M' = (W', R', \Vdash')$ とすると補題\ref{lem:simple_extension}より $M',1 \nvDash A$ でもある.
    $M'$ に対してのSolovay文 $\beta_0, \beta_1, \dots, \beta_n$ を構成し,$*$ はSolovay解釈とする.
    補題\ref{lem:solovay_interpretation}より $T \vdash \beta_1 \to \lnot A^*$ を得て,対偶を取れば $T \vdash A^* \to \beta_1$ が得られる.

    今,仮に $T \vdash A^*$ とすると議論が破綻することを示す.
    
    先ほど得た論理式より $T \vdash \beta_1$ であり,更に導出可能性条件 D1 より $T \vdash \Provable_T(\ulcorner \beta_1 \urcorner)$ が得られる.
    一方,単純拡張なので $0 R' 1$ であるからdより $T \vdash \beta_0 \to \lnot \Provable_T(\ulcorner \beta_1 \urcorner)$ であり,
    対偶を取れば $T \vdash \Provable_T(\ulcorner \beta_1 \urcorner) \to \lnot \beta_0$ が得られる.
    これらを合わせると $T \vdash \lnot \beta_0$ が得られる.
    したがって a より $T \vdash \bigvee_{i \in W \setminus \{0\}} \beta_i$ である. 

    $i \in W'$ かつ $i \neq 0$ としたとき,$d(i) \leq d(1)$ である.したがって補題 \ref{lem:cannot_access}より
    $M',i \vDash \Box^{d(1) + 1} \bot$ が得られる.
    更に補題\ref{lem:solovay_interpretation}より $T \vdash \beta_i \to (\Box^{d(1) + 1} \bot)^*$ であり,
    これはすなわち $T \vdash \beta_{i} \to \Provable^{d(1) + 1}_T(\ulcorner \bot \urcorner)$ である.

    $T \vdash \bigvee_{i \in W \setminus \{0\}} \beta_i$ と合わせると $T \vdash \Provable^{d(1) + 1}_T(\ulcorner \bot \urcorner)$ を得る.
    $\Provable^{d(1) + 1}_T(\ulcorner \bot \urcorner)$ は $\Sigma_1$ 文であり,$\Sigma_1$ 健全性より $T \vdash \bot$ が言える.
    しかしこれはおかしい.よって仮定がおかしく,$T \nvdash A^*$ である.
    \qed
\end{proof}

残ったのはSolovay文の列が構成可能であることを示すことである.

\printbibliography

\end{document}